% !TeX spellcheck = de_DE
%\input{header.tex}

\documentclass[10pt,a4paper]{article}
\usepackage[utf8]{inputenc}
\usepackage[ngerman]{babel}
\usepackage{amsmath}
\usepackage{amsfonts}
\usepackage{amssymb}
\usepackage{marvosym}
\usepackage{ gensymb }
%\usepackage{graphicx} % Allows for eps images
\usepackage{hyperref} %Links
\usepackage{url}
\renewcommand{\UrlFont}{\small\tt} %smaller font for urls
\usepackage[a4paper,lmargin={3cm},rmargin={3cm},tmargin={3cm},bmargin = {3cm}]{geometry}


% Literaturangabe
%\usepackage{biblatex}
%\bibliographystyle{plain}
\bibliographystyle{alpha} %http://sites.stat.psu.edu/~surajit/present/bib.htm
%\usepackage{natbib}
%\bibliographystyle{plainnat}

\usepackage{verbatim} %Fuer mehrzeilige kommentare \begin{comment} \end{comment}

%\begin{comment}
\usepackage{titlesec}
\titleformat{\subsection}[runin]% runin puts it in the same paragraph
       {\normalfont\bfseries}% formatting commands to apply to the whole heading
       {\thesubsection}% the label and number
       {0.5em}% space between label/number and subsection title
       {}% formatting commands applied just to subsection title
       []% e.g. [.] punctuation or other commands following subsection title\frac{•}{•}
%\end{comment}

%\usepackage{pdfpages} %For including PDFs
\usepackage{tikz} %For including Graphics generated by QtiPlot
%\usepackage{pdfcomment} %For generating comments in PDFs, e.g. \pdfsquarecomment{Kommentar}

%Some other shit
\usepackage{float}
\floatstyle{boxed}
\restylefloat{figure}
\usepackage{wrapfig}

\linespread{1.1}

\addto\captionsgerman{ \renewcommand{\figurename}{\small{\textbf{Abb.}}}
\captionsetup{figurewithin = section}
\captionsetup{font=small, labelfont=bf} }



\begin{document}

\pagenumbering{Roman}

%Metadaten
\title{Fortgeschrittenenpraktikum\\ -\\ Holographie }
%\date{02.02.2015}
\author{Leon Katzenmeier \\ René Vollmer}

\maketitle
%\vfill
\newpage

\tableofcontents
\vfill
\newpage

\pagenumbering{arabic}

\section{Dies ist eine Änderung von Leon um 9:11}
\section{Hinweise zum Protokoll}

Zur erfolgreichen Durchführung eines Praktikumsversuchs gehört die Anfertigung eines Versuchsprotokolls, in dem alle durchgeführten Versuche dokumentiert und die Ergebnisse dargestellt werden. Genau wie der Seminarvortrag sollte dieses auch für jemanden verständlich sein, der den Versuch nicht selbst durchgeführt hat. Es ist ratsam, schon zu Beginn des Experimentierens die Ausführung des Protokolls mit dem betreuenden Assistenten zu erörtern.\\
\\
Die Abgabe des Protokolls muss spätestens am Montag 14 Tage nach Versuchsbeginn erfolgen.\\
\\
Das Protokoll sollte sich auf das Wesentliche beschränken und nur die Informationen beinhalten, die für das Verständnis der Experimente und für die Auswertung notwendig sind. Es sollte darauf verzichtet werden, lange Abhandlungen über Themen zu verfassen, die man in Lehrbüchern finden kann. Alle Materialien, die aus anderen Arbeiten und/oder Büchern etc. übernommen werden, müssen entsprechend zitiert werden. Die Betreuer der Versuche geben dazu gerne Hinweise.\\
\\
Darüber hinaus mögen folgende Empfehlungen bei der Erstellung der Ausarbeitung nützlich sein:\\
\begin{enumerate}
\item Beginnen Sie umgehend mit dem Protokoll (keine Zettelwirtschaft!).
\item Bedienen Sie sich kurzer und klarer Ausdrucksweise.
\item Behandeln Sie den Stoff nur dort im Detail, wo es für die Durchführung Ihres Experiments wesentlich ist.
\item Lassen Sie den Umfang Ihres Berichts nicht beträchtlich über etwa 10 bis 15 Seiten hinauswachsen (ohne Anlagen).
\item Abbildungen und Tabellen durchnummerieren und mit Unterschriften versehen. Im Text darauf Bezug nehmen. Bei den Abbildungen sind die Achsen zu beschriften. 
\end{enumerate}

Bei den meisten Versuchen dürfte folgende Gliederung zu gebrauchen sein:

\begin{enumerate}

    \item{Inhaltsverzeichnis mit Seitenangaben}
    \item{Einleitung}\\
    Aufgabenstellung, Ziel der Messungen
    \item{Theoretische Grundlagen}\\
    Kurze Zusammenfassung der wichtigsten Überlegungen (Formeln), die zum Verständnis des Versuches notwendig sind\\
    Für Details: Hinweise auf die Literatur
    \item{Experimenteller Aufbau, Messmethode und Durchführung}\\
    Prinzip der Messmethode, Einzelheiten der Apparatur (nur soweit sie spezifisch und zur Beurteilung der Qualität der Messergebnisse von Bedeutung sind)
    \item{Messergebnisse mit Fehlerdiskussion}\\
    Im Protokoll nicht unbedingt alle Messwerte einzeln aufführen, sondern nur die zur Berechnung der Ergebnisse wesentlichen Daten\\
    Weitere Messwerte im Anhang
    \item{Diskussion der physikalischen Ergebnisse}\\
    Vergleich mit den theoretischen Erwartungen, evtl. auch mit Ergebnissen aus der Literatur
    \item{Zusammenfassung}
    \item{Literatur}\\
    Im Text herangezogene Quellen dort markieren und am Ende zusammenstellen
   \item{Anhang}\\
    Hierhin gehören in der Regel Originalmessprotokolle (wichtig) etc. 
    
\end{enumerate}

\textit{Entnommen von der Website der Uni Hamburg \cite{hinweise}.}

\section{Einleitung}
\subsection{Laser} Laser sind dolle Dinger!

\subsection{Formel 1} Tolle Formel $ h \rightarrow Z^0 Z^{*0} \rightarrow e^+ e^- \mu^+ \mu^- $ total inline.

\subsection{Formel 2} Wichtige Formel:
$$ a^2 = b^2 + c^2 $$
und Nummerierte Formel:
\begin{equation}
sin^2(x)+cos^2(x) = 1
\end{equation}


\subsection{Grafik}
\begin{figure}[h]
\centering
\input{img/Graph2.tex}

\caption[Parabel]{Tolle Grafik! Sowas wie expotentiell?}
\end{figure}
Mehr zu dem Unterschied zwischen \textit{input} und \textit{include} gibt es \href{http://de.wikibooks.org/wiki/LaTeX-W%C3%B6rterbuch:_include}{hier}



\section{Theoretische Grundlagen}
\section{Experimenteller Aufbau, Messmethode und Durchführung}
\section{Messergebnisse mit Fehlerdiskussion}
\section{Diskussion der physikalischen Ergebnisse}


\section{Zusammenfassung}
\subsection{}  Das will ich belegen \cite[p. 2]{erstesBuch}



\newpage

%\pagenumbering{Roman}
\pagenumbering{Alph}


%\nocite{zweitesBuch}
\bibliography{praktikum.bib}
\addcontentsline{toc}{section}{Literatur}

\newpage

\listoffigures %lof
\addcontentsline{toc}{section}{Abbildungsverzeichnis}
%\tableofcontents %toc
%\listoftables %lot

\end{document} 