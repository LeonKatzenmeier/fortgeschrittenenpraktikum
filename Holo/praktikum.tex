% !TeX spellcheck = de_DE
%\input{header.tex}

\documentclass[10pt,a4paper]{article}
\usepackage[utf8]{inputenc}
\usepackage[ngerman]{babel}
\usepackage{amsmath}
\usepackage{amsfonts}
\usepackage{amssymb}
\usepackage{marvosym}
\usepackage{ gensymb }
%\usepackage{graphicx} % Allows for eps images
\usepackage{hyperref} %Links
\usepackage[a4paper,lmargin={3cm},rmargin={3cm},tmargin={3cm},bmargin = {3cm}]{geometry}


% Literaturangabe
%\usepackage{biblatex}
%\usepackage{cite} %Erlaubt verweise auf Literatur

\usepackage{verbatim} %Fuer mehrzeilige kommentare \begin{comment} \end{comment}

%\begin{comment}
\usepackage{titlesec}
\titleformat{\subsection}[runin]% runin puts it in the same paragraph
       {\normalfont\bfseries}% formatting commands to apply to the whole heading
       {\thesubsection}% the label and number
       {0.5em}% space between label/number and subsection title
       {}% formatting commands applied just to subsection title
       []% e.g. [.] punctuation or other commands following subsection title\frac{•}{•}
%\end{comment}

%\usepackage{pdfpages} %For including PDFs
\usepackage{tikz} %For including Graphics generated by QtiPlot
%\usepackage{pdfcomment} %For generating comments in PDFs, e.g. \pdfsquarecomment{Kommentar}

%Some other shit
\usepackage{float}
\floatstyle{boxed}
\restylefloat{figure}
\usepackage{wrapfig}

\linespread{1.1}

\addto\captionsgerman{ \renewcommand{\figurename}{\small{\textbf{Abb.}}}
\captionsetup{figurewithin = section}
\captionsetup{font=small, labelfont=bf} }



\begin{document}

\pagenumbering{Roman}

%Metadaten
\title{Fortgeschrittenenpraktikum\\ -\\ Holographie }
%\date{02.02.2015}
\author{Leon Katzenmeier \\ René Vollmer}

\maketitle
%\vfill
\newpage

\tableofcontents
\vfill
\newpage

\pagenumbering{arabic}

\section{Theorie}
\subsection{Laser} Laser sind dolle Dinger!

\subsection{Grafik}
\begin{figure}[h!]
\centering
\input{img/Graph2.tex}

\caption{Tolle Grafik!}
\end{figure}


\subsection{Formel 1} Tolle Formel $ h \rightarrow Z^0 Z^{*0} \rightarrow e^+ e^- \mu^+ \mu^- $ total inline.

\subsection{Formel 2} Wichtige Formel:
$$ a^2 = b^2 + c^2 $$

\subsection{Formel 3} Nummerierte Formel:
\begin{equation}
sin^2(x)+cos^2(x) = 1
\end{equation}


\section{Versuchsdurchführung}


\section{Auswertung}
\subsection{}  Das will ich belegen \cite[p. 2]{erstesBuch}
\subsection{} Das auch  \cite{link}



\newpage

%\pagenumbering{Roman}
\pagenumbering{Alph}


\nocite{zweitesBuch}
\bibliography{praktikum.bib}
\bibliographystyle{plain}

\newpage
\addcontentsline{toc}{section}{Abbildungsverzeichnis}
\listoffigures


\end{document} 